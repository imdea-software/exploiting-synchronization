\documentclass{acm_proc_article-sp}

\usepackage[utf8]{inputenc}
\usepackage{microtype}
\usepackage{xspace}
\usepackage{hyperref}
\usepackage{breakurl}
\def\ttat{\mtt{@}} % the at package clobbers this
\usepackage{at}
\usepackage{url}
\usepackage{amsmath}
\usepackage{amsfonts}
\usepackage{amssymb}
\usepackage{latexsym}
\usepackage{wasysym} % causing problems with llncs's bold vectors
\usepackage{stmaryrd}
\usepackage[figure,lined]{algorithm2e}
\usepackage[ligature,reserved,inference]{semantic}
\usepackage{enumerate}
\usepackage{enumitem}
\usepackage{listings}
\usepackage{mathpartir}
\usepackage{graphicx}
\usepackage{array}
% \usepackage{multicol}


 % added
\usepackage{syntax}

\renewcommand{\ttdefault}{pxtt}
\lstset{tabsize=2}

\pagestyle{plain}
\bibliographystyle{splncs}

\title{Exploiting Synchronization in the Analysis of Shared-Memory Asynchronous Programs}

% Synchronization-Aware Analysis of Asynchronous Programs

\numberofauthors{3}

\author{
  \alignauthor
  Michael Emmi \\
  \affaddr{IMDEA Software Institute} \\
  \email{michael.emmi@imdea.org}
  \alignauthor
  Burcu Kulahcioglu Ozkan \\
  \affaddr{Koç University} \\
  \email{bkulahcioglu@ku.edu.tr}
  \alignauthor
  Serdar Tasiran \\
  \affaddr{Koç University} \\
  \email{stasiran@ku.edu.tr}
}

\input macros
\newtheorem{theorem}{Theorem}
\newtheorem{lemma}{Lemma}

\begin{document}
  \maketitle

  \begin{abstract}

    As asynchronous programming becomes more mainstream, program analyses
    capable of automatically uncovering programming errors are increasingly in
    demand. Since asynchronous program analysis is computationally costly,
    current approaches sacrifice completeness and focus on limited sets of
    asynchronous task schedules that are likely to expose programming errors.
    These approaches are based on \emph{parameterized} task schedulers, each
    which admits schedules which are variations of a default deterministic
    schedule. By increasing the parameter value, a larger variety of schedules
    is explored, at a higher cost. The efficacy of these approaches depends
    largely on the default deterministic scheduler on which varying
    schedules are fashioned.
    
    We find that the limited exploration of asynchronous program behaviors can
    be made more efficient by designing parameterized schedulers which better
    match the inherent ordering of program events, e.g.,~arising from waiting
    for an asynchronous task to complete. We follow a reduction-based
    ``sequentialization'' approach to analyzing asynchronous programs, which
    leverages existing (sequential) program analysis tools by encoding
    asynchronous program executions, according to a particular scheduler, as
    the executions of a sequential program. Analysis based on our new scheduler
    comes at no greater computational cost, and provides strictly greater
    behavioral coverage than analysis based on existing parameterized
    schedulers; we validate these claims both conceptually, with complexity and
    behavioral-inclusion arguments, and empirically, by discovering actual
    reported bugs faster with smaller parameter values.
    
  \end{abstract}
  
  \input intro
  \input programs
  \input schedulers
  \input compositional
  \input seq
  \input complexity
  \input exp
  \input related
  
  % TODO remove this line
  \nocite{*}

  \bibliography{biblio}
  
  \newpage
  \appendix
  \input programs-appendix
  \input proofs
  \input bugs
  
\end{document}