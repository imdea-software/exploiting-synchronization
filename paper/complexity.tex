\section{Complexity}
\label{sec:complexity}

While Section~\ref{sec:schedulers} establishes that $\dfw(K)$ generally
reaches more program variable valuations than $\df(K)$ does, an obvious
concern would be the cost at which it does so. In this section we demonstrate
that despite the increased power of $\dfw(K)$ with respect to
reachability, the essential cost of exploration is roughly equivalent, in that
the reachability problem falls into the same NP-complete class as that of
$\df(K)$. As is standard in the literature, we focus on the effects on
complexity arising from concurrency, factoring out effects arising from data;
we thus measure the asymptotic complexity of the global-variable value
reachability problem assuming program variables have finite domains, and that
the number of program variables is fixed. Otherwise, general infinite data
domains would lead to undecidability, and the exponential number of valuations
of a non-fixed number of program variables would interfere with our complexity
measurement. Formally, the \emph{$\dfw(K)$ reachability problem}
asks whether a given global program variable valuation $g$ of a given program
$P$ is included in $R(P,\dfw(K))$, for a given $K \in \<Nats>$, written
in unary.

NP-hardness follows directly from the NP-hardness of $\df(K)$'s
reachability problem~\cite{conf/popl/EmmiQR11}, since $R(P,\df(K)) =
R(P,\dfw(K))$ for programs $P$ without \lstinline{wait} statements.

\begin{lemma}
  \label{lem:np:hard}
  $\dfw(K)$ reachability is NP-hard.
\end{lemma}

Our proof of NP-membership reduces the problem to reachability in
sequential programs with a fixed number of variables in $K$. While this amounts
to a sort of sequentialization, our sequentialization of Section~\ref{sec:seq}
is inadequate, since $@S(P,K)$ has a \emph{linear} number of program variables
in $K$, evaluating to an exponential number of valuations in $K$. The crux of
our proof is thus to design a sequentialization which uses only a
\emph{constant} number of additional program variables, independently of $K$.

\begin{lemma}
  \label{lem:np:mem}
  $\dfw(K)$ reachability is in NP.
\end{lemma}

Combining lemmas, we have a tight complexity result.

\begin{theorem}
  \label{thm:np}
  $\dfw(K)$ reachability is NP-complete.
\end{theorem}
