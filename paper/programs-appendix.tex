\section{Sequential Semantics}
\label{sec:programs:appendix}

Transition rules for the standard sequential program statements in an asynchronous program are listed in Figure ~\ref{fig:semantics:seq}. 

The {\sc Skip} rule just proceeds to the next statement. The {\sc Assume} rule proceeds only if the expression \emph{e} evaluates to true. 
The {\sc Then} rule proceed to the then branch if the current valuation of the given expression
\emph{e} evaluates to true, {\sc Else} rule proceeds to the else branch if \emph{e} evaluates to false. The {\sc Loop} rule iterates the loop if the given expression \emph{e} evaluates to true, and {\sc Break} skips the loop if \emph{e} evaluates to false. The {\sc Global}/{\sc Local} statements set the value of a global/local variable to the value that given expression evaluates to. 

\begin{figure*}[t]
  \centering
  \begin{mathpar}
    
    \inferrule[Skip]{
    }{
      C[\<skip>; s] -> C[s]
    }
    
    \inferrule[Assume]{
      \<true> \in e(C)
    }{
      C[\<assume> e] -> C[\<skip>]
    }
    
    \inferrule[Then]{
      \<true> \in e(C)
    }{
      C[\<if> e \<then> s_1 \<else> s_2] -> C[s_1]
    }
    
    \inferrule[Else]{
      \<false> \in e(C)
    }{
      C[\<if> e \<then> s_1 \<else> s_2] -> C[s_2]
    }
    
    \inferrule[Loop]{
      \<true> \in e(C)
    }{
      C[\<while> e \<do> s] -> C[s; \<while> e \<do> s]
    }
    
    \inferrule[Break]{
      \<false> \in e(C)
    }{
      C[\<while> e \<do> s] -> C[\<skip>]
    }
    
    \inferrule[Global]{
      x \text{ is a global variable} \\
      v \in e(g,M)
    }{
      \tup{g, M[x := e]} -> \tup{g(x |-> v), M[\<skip>]}
    }
    
    \inferrule[Local]{
      x \text{ is a local variable} \\\\
      v \in e(g,@l) \\ @l_2 = @l_1 (x |-> v)
    }{
      \tup{g, m \U \mset{i,\tup{@l_1,S[x := e]} \cdot w, v}}
      -> \tup{g, m \U \mset{i, \tup{@l_2,S[\<skip>]} \cdot w, v}}
    }

  \end{mathpar}
  \caption{Transition rules for sequential statements.}
  \label{fig:semantics:seq}
\end{figure*}