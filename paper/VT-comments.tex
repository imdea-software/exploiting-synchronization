\documentclass{llncs}
\usepackage[utf8]{inputenc}
\usepackage{microtype}
\usepackage{xspace}
\usepackage{hyperref}
\usepackage{breakurl}
\def\ttat{\mtt{@}} % the at package clobbers this
\usepackage{at}
\usepackage{url}
\usepackage{amsmath}
\usepackage{amsfonts}
\usepackage{amssymb}
\usepackage{latexsym}
\usepackage{wasysym} % causing problems with llncs's bold vectors
\usepackage{stmaryrd}
\usepackage[figure,lined]{algorithm2e}
\usepackage[ligature,reserved,inference]{semantic}
\usepackage{enumerate}
\usepackage{enumitem}
\usepackage{listings}
\usepackage{mathpartir}
\usepackage{graphicx}
\usepackage{array}
% \usepackage{multicol}


 % added
\usepackage{syntax}

\renewcommand{\ttdefault}{pxtt}
\lstset{tabsize=2}

\pagestyle{plain}
\bibliographystyle{splncs}

\title{Synchronization-Aware Analysis of Asynchronous Programs}

\author{
  Michael Emmi\inst{2}
  \and Burcu Külahçıoğlu Özkan\inst{3}
  \and Serdar Tasiran\inst{3}
}

\institute{
  IMDEA Software Institute, Spain, \email{michael.emmi@imdea.org}
  \and Koç University, Turkey, \email{bkulahcioglu@ku.edu.tr},
  \email{stasiran@ku.edu.tr}
}

\input macros

\begin{document}
  \maketitle

  \begin{abstract}

    As asynchronous programming becomes more mainstream, program analyses which
%use "that" with restrictive clause (although this rule is not always strictly followed, it is better to use it, especially when consistently applying the rule will add to clarity as in the next para below. Here it might be better to use neither and say something like "program analyses able to automatically..."

    automatically uncover programming errors are increasingly in demand. Since
    asynchronous program analysis is rather challenging,%this is not useful. Presumably anything about which a paper is being written is/should be rather challenging. Try to describe these challenges in a way that emphasizes the signifacance of your proposed approach.
 current approaches
    sacrifice completeness and focus on a limited set of thread schedules that
    are empirically likely to expose programming errors. The consideration of
    effective \emph{parameterized} thread schedulers, which increasingly allow
    more diverse%unclear -- more diverse threads to interact, or more interaction of diverse threads? Presumably an expert reader would be able to guess, but it is ambiguous as written
 thread interactions at higher cost as their parameters are
    increased, is essential to such approaches. The effectiveness of these
    approaches depends critically on a default (deterministic) scheduler, on
    which varying schedules are fashioned by increasing nondeterminism with
    increasing parameter values.%Again, an expert reader would know, but the writing is ambiguous. Does this mean that nondeterminism increases AS parameter values increase or that nondeterminism is increased BY increasing the parameter values?

    We find that the limited exploration of relevant asynchronous program
    behaviors can be made more efficient by designing parameterized schedulers
    which%again, this is a restrictive clause, so use that for grammar and clarity
 correspond well with the intended synchronization and ordering of
    program events, e.g.%is the following an example of a program event? 
 arising from waiting for an asynchronous task to
    complete. We follow a reduction-based ``sequentialization'' approach to
    analyzing asynchronous programs, which%this is why it is important to use that consistently for restrictive clauses and which for non-restrictive. It is unclear here if you are talking about a general characteristic of this approach (non-restrictive, which) or limiting the scope of the discussion to only those types of this approach with this characteristic (restrictive, that)
 can leverage existing (sequential)
    program analysis tools,%no comma needed here
 by encoding the program executions according to a
    ``synchronization-aware'' scheduler as executions of a sequential program.
    Analysis based on our new scheduler comes at no greater computational cost,
    and provides strictly greater %is this a standard phrase? if not, unclear/imprecise
behavioral coverage, than analysis based on
    existing parameterized schedulers; we validate this%what? You have made two claims above. Avoid unclear referents by using this + ______
 both conceptually, with
    complexity and behavioral-inclusion arguments, and empirically, by
    discovering actual reported bugs faster with smaller parameter values.
    
  \end{abstract}
  
  \input intro
  \input programs
  \input schedulers
  \input compositional
  \input seq
  \input complexity
  \input exp
  \input related
  
  % TODO remove this line
  \nocite{*}

  \bibliography{biblio}
  
  % \newpage
  \appendix
  \input programs-appendix
  \input bugs
  \input exp-appendix
  \input proofs
  
\end{document}